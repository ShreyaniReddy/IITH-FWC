\documentclass[10pt,twocolumn]{article}
\usepackage[margin=0.5in]{geometry}
\usepackage[cmex10]{amsmath}
\usepackage{amsmath}
\usepackage{amsmath,amssymb,amsfonts}
\usepackage{graphicx}
\usepackage{textcomp}
\usepackage{amsmath,amssymb,amsfonts,amsthm}
\usepackage{gensymb}
\newcommand*{\Comb}[2]{{}^{#1}C_{#2}}%
\let\vec\mathbf
\title{
Probability Assignment
}
\author{GINNA SHREYANI}
\date{}
\providecommand{\pr}[1]{\ensuremath{\Pr\left(#1\right)}}


\begin{document}
\maketitle

\textbf{12.13.3.3}\\
Of the students in a college, it is known that 60\% reside in hostel and 40\% are day scholars (not residing in hostel). Previous year results report that 30\% of all students who reside in hostel attain A grade and 20\% of day scholars attain A grade in their annual examination. At the end of the year, one student is chosen at random from the college and he has an A grade, what is the probability that the student is a hostlier?
\subsection*{Solution}
Using Baye's Rule:\\
Let the probability of students living in hostel be $\Pr(H=1)$, therefore the students who are day scholars can be given as $\Pr(H=0)$\\
The probability of the students getting grade A is given as $\Pr(A=1)$.\\
By given information,
\begin{align}
	\Pr(H=1) &= \frac{60}{100}\\
	\Pr(H=0) &= \frac{40}{100}\\
	\Pr(A=1|H=1) &= \frac{30}{100}\\
	\Pr(A=1|H=0) &= \frac{20}{100}
\end{align}
Thus,
\begin{align}
	\Pr(A=1) &= \sum_{i=0}^1 \Pr(A = 1|H = i)\Pr(H=i)\\
	\text{$\Pr(A=1$)}
	&=\Pr(A = 1|H = 0)\Pr(H = 0)+\\
	&\qquad \Pr(A = 1|H = 1)\Pr(H = 1)\\
	\Pr(A=1) &= \left(\frac{20}{100} \times \frac{40}{100}\right)+\left(\frac{30}{100} \times \frac{60}{100}\right)\\
	\Pr(A=1) &= \frac{26}{100}\\
	\Pr(H=1|A=1) &= \frac{\Pr(A=1|H=1)\Pr(H=1)}{\Pr(A=0)}\\
	\Pr(H=1|A=1) &= \frac{9}{13}
\end{align}
The probability that the student is a hostlier who has A grade is $\frac{9}{13}$.
\end{document}

