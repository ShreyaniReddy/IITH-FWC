\documentclass[journal,10pt,twocolumn]{article}
\usepackage{graphicx}
\usepackage[margin=0.5in]{geometry}
\usepackage[cmex10]{amsmath}
\usepackage{array}
\usepackage{booktabs}
\usepackage{mathtools}
\title{\textbf{Optimization Assignment - 2}}
\author{Ginna Shreyani}
\date{November 2022}


\providecommand{\norm}[1]{\left\lVert#1\right\rVert}
\providecommand{\abs}[1]{\left\vert#1\right\vert}
\let\vec\mathbf
\newcommand{\myvec}[1]{\ensuremath{\begin{pmatrix}#1\end{pmatrix}}}
\newcommand{\mydet}[1]{\ensuremath{\begin{vmatrix}#1\end{vmatrix}}}
\providecommand{\brak}[1]{\ensuremath{\left(#1\right)}}
\providecommand{\lbrak}[1]{\ensuremath{\left(#1\right.}}
\providecommand{\rbrak}[1]{\ensuremath{\left.#1\right)}}
\providecommand{\sbrak}[1]{\ensuremath{{}\left[#1\right]}}

\begin{document}

\maketitle
\section{Problem Statement}
Show that if ABC is a triangle, and P any point then $\vec{(PA)}^2+\vec{(PB)}^2+\vec{(PC)}^2$ will be minimum when $\vec{P}$ is at the centroid.\\
\section{Solution:} 
Given, ABC is a triangle and $\vec{P}$ be any point, then 
\begin{align}
	&\vec{(P-A)}^2+\vec{(P-B)}^2+\vec{(P-C)}^2
\label{eq-1}
\end{align}
should be minimum when P is centroid.\\
For ($\ref{eq-1}$) to be minimum the differentiation of ($\ref{eq-1}$) should be equated to zero.\\
\begin{align}
	&\frac{d}{d(x,y)}[\vec{(P-A)}^2+\vec{(P-B)}^2+\vec{(P-C)}^2]=0
\label{eq-2}
\end{align}
By solving the ($\ref{eq-2}$), we get
\begin{multline}
	2\vec{(P-A)}\frac{d}{d(P)}[(P-A)]+2\vec{(P-B)}\frac{d}{d(P)}[(P-B)]+2\vec{(P-C)}\frac{d}{d(P)}[(P-C)] = 0\\
\end{multline}
\begin{align*}
	&2(\vec{(P-A)}+\vec{(P-B)}+\vec{(P-C)})=0\\
	&\vec{P} = \frac{\vec{A}+\vec{B}+\vec{C}}{3}
\end{align*}
Therefore, minimum of ($\ref{eq-1}$) is obtained when P is the centroid of the triangle ABC.\\
Using cvxpy method,\\
Let us take the vertices of the triangle,\\
\begin{align*}
	&\vec{A}=\myvec{0\\0}\\
	&\vec{B}=\myvec{4\\0}\\
	&\vec{C}=\myvec{3\\4}\\
\end{align*}
The minimum value is given as,\\
\begin{align*}
	&\min_{\vec{P}}(\vec{(P-A)^2}+\vec{(P-B)^2}+\vec{(P-C)^2})
\end{align*}
The minimum is obtained by equating the above equation to 0.\\
\begin{align*}
	&6\vec{P} = 2\vec{(A+B+C)}\\
	&\vec{P} =\frac{1}{3}\vec{(A+B+C)}\\
	&\vec{P} = \frac{1}{3}(\myvec{0\\0}+\myvec{4\\0}+\myvec{3\\4})\\
	&\vec{P} =\myvec{\frac{7}{3}\\\frac{4}{3}}
\end{align*}
\end{document}
